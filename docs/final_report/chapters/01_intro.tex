\chapter{Mở đầu}

% Trình bày lí do chọn đề tài, mục đích, đối tượng và phạm vi nghiên cứu

\section{Lý do chọn đề tài}

Sinh viên Đại học Công nghệ Thông tin (UIT) phải quản lý một lượng lớn thông tin học vụ từ nhiều hệ thống khác nhau.
Các thông tin này bao gồm quy định về điều kiện tốt nghiệp, yêu cầu môn học theo ngành, lịch học, lịch thi, điểm số,
và các tài liệu học tập. Tuy nhiên, các thông tin này không được tập trung ở một nơi mà phân tán trên nhiều nền tảng
và hệ thống khác nhau như DAA portal, website khoa, trang thông báo, v.v.

Để tìm kiếm thông tin, sinh viên phải truy cập nhiều trang web, ghi nhớ các quy định, hoặc liên hệ với cán bộ hỗ trợ.
Quá trình này tốn thời gian, dễ dẫn đến nhầm lẫn, và không hiệu quả. Nhận thấy vấn đề này, đề tài quyết định xây dựng
một AI agent có khả năng trả lời các câu hỏi tự nhiên từ sinh viên, giúp họ nhanh chóng truy cập thông tin học vụ một cách
dễ dàng và chính xác.

\section{Mục đích}

Đề tài có mục đích xây dựng một AI agent có khả năng hỗ trợ sinh viên UIT trong việc quản lý và tra cứu thông tin học vụ.
Cụ thể, AI agent sẽ có khả năng hiểu các câu hỏi tự nhiên từ sinh viên, suy luận logic để xác định loại thông tin cần tìm,
gọi các công cụ thích hợp (truy vấn knowledge base hoặc lấy dữ liệu từ DAA portal), và cung cấp câu trả lời chính xác
một cách nhanh chóng.

Bên cạnh đó, đề tài cũng nhằm chứng minh khả năng tích hợp và ứng dụng thực tế các công nghệ AI hiện đại như Retrieval-Augmented
Generation (RAG), Model Context Protocol (MCP), và LangGraph trong việc xây dựng các ứng dụng phục vụ người dùng cuối.

\section{Đối tượng nghiên cứu}

Đối tượng nghiên cứu của đề tài là các sinh viên Đại học Công nghệ Thông tin (UIT), đặc biệt là những sinh viên
cần tra cứu thông tin học vụ như quy định tốt nghiệp, yêu cầu môn học, lịch thi, điểm số, thời khóa biểu, và tài liệu
học tập. Ngoài ra, đề tài cũng tập trung vào việc phát triển một AI agent có khả năng hiểu và xử lý các câu hỏi tự nhiên
từ sinh viên, làm nền tảng để các ứng dụng tương tự có thể áp dụng cho các trường đại học khác.

\section{Phạm vi nghiên cứu}

Phạm vi nghiên cứu của đề tài bao gồm:

\textbf{Về mặt chức năng:} AI agent được xây dựng để trả lời các câu hỏi liên quan đến (1) quy định đào tạo và tốt nghiệp
của UIT, (2) yêu cầu môn học theo từng ngành học, (3) thông tin lịch thi và lịch học, (4) điểm số và kết quả học tập
lấy từ DAA portal. Đề tài không bao gồm các chức năng khác như đăng ký môn học, xin học lại, hay các dịch vụ hành chính
khác.

\textbf{Về mặt công nghệ:} Đề tài tập trung vào việc tích hợp ba thành phần chính: Retrieval-Augmented Generation (RAG)
cho knowledge base, Model Context Protocol (MCP) cho tích hợp công cụ, và LangGraph cho orchestration logic của AI agent.
Phạm vi không bao gồm việc huấn luyện mô hình LLM từ đầu, mà sử dụng các mô hình có sẵn như Claude hoặc các mô hình mã
nguồn mở khác.

\textbf{Về mặt triển khai:} Hệ thống được thiết kế cho môi trường UIT với dữ liệu và hệ thống của trường này. Tuy nhiên,
kiến trúc được thiết kế để có thể mở rộng cho các trường đại học khác trong tương lai.
